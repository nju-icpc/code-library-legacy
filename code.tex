\section{比赛配置}
\subsection{代码库校验和}
\createlinenumber{1}{01b4}
\createlinenumber{2}{44f9}
\createlinenumber{3}{4de6}
\createlinenumber{4}{c502}
\createlinenumber{5}{427e}
\createlinenumber{6}{b41f}
\createlinenumber{7}{d74e}
\createlinenumber{8}{427e}
\createlinenumber{9}{f7db}
\createlinenumber{10}{f335}
\begin{lstlisting}[language=Python]
# 代码库校验用于检查代码库录入是否正确,忽略每行的空白字符和注释(//)
# 使用方法: python checksum.py < 1001.cpp
# 输出: 每一行代码及其校验和(md5)
import re, sys, hashlib

def digest_line(s):
	return hashlib.md5(re.sub(r'\s|//.*', '', s)).hexdigest()[-4:]

for line in sys.stdin.read().strip().split("\n"):
	print digest_line(line), line
\end{lstlisting}
\subsection{vim配置文件}
\createlinenumber{11}{44ed}
\createlinenumber{12}{914c}
\createlinenumber{13}{7db5}
\createlinenumber{14}{57b2}
\createlinenumber{15}{9832}
\createlinenumber{16}{e416}
\createlinenumber{17}{7232}
\createlinenumber{18}{740c}
\createlinenumber{19}{5913}
\begin{lstlisting}[language={}]
# vimrc 配置文件
set nocompatible
set number
set ruler
set showcmd
set autoindent
set cindent
set smartindent
set shiftwidth=4
\end{lstlisting}

\section{其他}
\subsection{校赛Meeting标程}
\createlinenumber{20}{1915}
\createlinenumber{21}{1fa4}
\createlinenumber{22}{427e}
\createlinenumber{23}{842a}
\createlinenumber{24}{8f9d}
\createlinenumber{25}{427e}
\createlinenumber{26}{4d45}
\createlinenumber{27}{4506}
\createlinenumber{28}{ec32}
\createlinenumber{29}{e2fe}
\createlinenumber{30}{3df7}
\createlinenumber{31}{b00a}
\createlinenumber{32}{05ee}
\createlinenumber{33}{4506}
\createlinenumber{34}{5c82}
\createlinenumber{35}{c0ce}
\createlinenumber{36}{583c}
\createlinenumber{37}{95cf}
\createlinenumber{38}{547e}
\createlinenumber{39}{95cf}
\createlinenumber{40}{427e}
\createlinenumber{41}{afdd}
\createlinenumber{42}{4506}
\createlinenumber{43}{918b}
\createlinenumber{44}{b4a4}
\createlinenumber{45}{4506}
\createlinenumber{46}{3dd0}
\createlinenumber{47}{8a02}
\createlinenumber{48}{40c0}
\createlinenumber{49}{95cf}
\createlinenumber{50}{39f8}
\createlinenumber{51}{95cf}
\createlinenumber{52}{427e}
\createlinenumber{53}{299c}
\createlinenumber{54}{4506}
\createlinenumber{55}{9523}
\createlinenumber{56}{918b}
\createlinenumber{57}{427e}
\createlinenumber{58}{ba03}
\createlinenumber{59}{4506}
\createlinenumber{60}{e322}
\createlinenumber{61}{6f5e}
\createlinenumber{62}{4506}
\createlinenumber{63}{3dd0}
\createlinenumber{64}{8a02}
\createlinenumber{65}{7091}
\createlinenumber{66}{95cf}
\createlinenumber{67}{a0d8}
\createlinenumber{68}{95cf}
\createlinenumber{69}{427e}
\createlinenumber{70}{7021}
\createlinenumber{71}{95cf}
\begin{lstlisting}[language=C++]
#include <stdio.h>
#include <math.h>

int u1, v1, u2, v2, u3, v3;
double x, y, d;

double f(double u, double v)
{
   double r=0;
   r+=sqrt((u-u1)*(u-u1)+(v-v1)*(v-v1));
   r+=sqrt((u-u2)*(u-u2)+(v-v2)*(v-v2));
   r+=sqrt((u-u3)*(u-u3)+(v-v3)*(v-v3));
   if(r<d)
   {
      d=r;
      x=u;
      y=v;
   }
   return r;
}

double f(double u)
{
   double L, R, M1, M2;
   for(L=-1001, R=1001; R-L>1e-6; )
   {
      M1=(L*2+R)/3;
      M2=(L+R*2)/3;
      f(u, M1)<f(u, M2)?(R=M2):(L=M1);
   }
   return f(u, (L+R)/2);
}

int main()
{
   int T;
   double L, R, M1, M2;
   
   for(scanf("%d", &T); T--; )
   {
      scanf("%d %d %d %d %d %d", &u1, &v1, &u2, &v2, &u3, &v3);
      for(d=1e100, L=-1001, R=1001; R-L>1e-6; )
      {
         M1=(L*2+R)/3;
         M2=(L+R*2)/3;
         f(M1)<f(M2)?(R=M2):(L=M1);
      }
      printf("(%.3lf,%.3lf)\n", x, y);
   }
   
   return 0;
}
\end{lstlisting}
