\section{比赛配置}
\subsection{代码库校验和}
\createlinenumber{1}{c502}
\createlinenumber{2}{427e}
\createlinenumber{3}{b41f}
\createlinenumber{4}{d74e}
\createlinenumber{5}{427e}
\createlinenumber{6}{f7db}
\createlinenumber{7}{f335}
\createlinenumber{8}{427e}
\begin{mdframed}[hidealllines=true,backgroundcolor=blue!5]
计算每一行忽略空白字符和\texttt{//}注释后内容的MD5 Hash。
使用\texttt{python checksum.py < code.cpp}打印校验和,并与代码库侧面的数值比对。
\end{mdframed}\vspace{-10pt}
\begin{lstlisting}[language=Python]
import re, sys, hashlib

def digest_line(s):
    return hashlib.md5(re.sub(r'\s|//.*', '', s)).hexdigest()[-4:]

for line in sys.stdin.read().strip().split("\n"):
    print digest_line(line), line

\end{lstlisting}
\subsection{Vim配置文件}
\createlinenumber{9}{914c}
\createlinenumber{10}{7db5}
\createlinenumber{11}{57b2}
\createlinenumber{12}{9832}
\createlinenumber{13}{e416}
\createlinenumber{14}{7232}
\createlinenumber{15}{740c}
\createlinenumber{16}{5913}
\createlinenumber{17}{427e}
\begin{lstlisting}[language={}]
set nocompatible
set number
set ruler
set showcmd
set autoindent
set cindent
set smartindent
set shiftwidth=4

\end{lstlisting}
\subsection{堆栈外挂}
\createlinenumber{18}{bebe}
\createlinenumber{19}{effc}
\createlinenumber{20}{4e99}
\createlinenumber{21}{427e}
\createlinenumber{22}{427e}
\createlinenumber{23}{7bc9}
\createlinenumber{24}{afcf}
\createlinenumber{25}{ac7a}
\createlinenumber{26}{a9ea}
\createlinenumber{27}{1937}
\createlinenumber{28}{427e}
\createlinenumber{29}{3117}
\createlinenumber{30}{427e}
\createlinenumber{31}{3750}
\createlinenumber{32}{427e}
\createlinenumber{33}{427e}
\createlinenumber{34}{6856}
\createlinenumber{35}{7021}
\createlinenumber{36}{95cf}
\createlinenumber{37}{427e}
\begin{mdframed}[hidealllines=true,backgroundcolor=blue!5]
将堆栈指针指向用户空间的缓存中从而获得无限大的堆栈。注意在返回前将堆栈指针恢复,否则将出现Segmentation Fault。
\end{mdframed}\vspace{-10pt}
\begin{lstlisting}[language=C++]
const int STK_SZ = 2000000; // 堆栈的格数
char STK[STK_SZ * sizeof(void*)];     
void *STK_BAK;

// 32/64位自动检测
#if defined(__i386__)
  define SP "%%esp"
#elif defined(__x86_64__)
  #define SP "%%rsp"
#endif

int main() {
  // 将堆栈指针移动到用户缓存
  asm volatile("mov " SP ",%0; mov %1," SP: "=g"(STK_BAK):"g"(STK+sizeof(STK)):);
  // 这时候就可以愉快地递归了!
  // 将堆栈指针恢复
  asm volatile("mov %0," SP::"g"(STK_BAK));
  return 0;
}

\end{lstlisting}
\subsection{I/O外挂}
\createlinenumber{38}{6540}
\createlinenumber{39}{eb51}
\createlinenumber{40}{4e24}
\createlinenumber{41}{d4df}
\createlinenumber{42}{2760}
\createlinenumber{43}{7181}
\createlinenumber{44}{5bd0}
\createlinenumber{45}{427e}
\createlinenumber{46}{3117}
\createlinenumber{47}{486b}
\createlinenumber{48}{b496}
\createlinenumber{49}{95cf}
\createlinenumber{50}{427e}
\createlinenumber{51}{427e}
\begin{mdframed}[hidealllines=true,backgroundcolor=blue!5]
快速整数输入输出,只支持非负整数。
\end{mdframed}\vspace{-10pt}
\begin{lstlisting}[language=C++]
#define BUFSIZE 20000000
char buf[BUFSIZE], *pt = buf;
#define scan(t) do { \
  int t = 0; \
  while (!((*pt) >= '0' && (*pt) <= '9')) pt ++; \
  while (((*pt) >= '0' && (*pt) <= '9')) t = t * 10 + (*(pt ++)) - '0'; \
} while (0)
 
int main() {
	fread(buf, 1, BUFSIZE, stdin);
	scan(N); scan(M); // ...
}


\end{lstlisting}
\subsection{编译优化外挂}
\createlinenumber{52}{de4b}
\createlinenumber{53}{427e}
\begin{mdframed}[hidealllines=true,backgroundcolor=blue!5]
开启GCC的O2编译优化。
\end{mdframed}\vspace{-10pt}
\begin{lstlisting}[language=C++]
#pragma GCC optimize("O2")

\end{lstlisting}
\section{组合优化}
\subsection{上下界最小费用可行流}
\createlinenumber{54}{193c}
\createlinenumber{55}{89b6}
\createlinenumber{56}{9cf5}
\createlinenumber{57}{bff2}
\createlinenumber{58}{5a99}
\createlinenumber{59}{0cc6}
\createlinenumber{60}{33d6}
\createlinenumber{61}{f4d6}
\createlinenumber{62}{815d}
\createlinenumber{63}{5053}
\createlinenumber{64}{4113}
\createlinenumber{65}{fc78}
\createlinenumber{66}{49d4}
\createlinenumber{67}{0f35}
\createlinenumber{68}{00e3}
\createlinenumber{69}{915b}
\createlinenumber{70}{e6b0}
\createlinenumber{71}{7c4f}
\createlinenumber{72}{b5bb}
\createlinenumber{73}{49d4}
\createlinenumber{74}{8e08}
\createlinenumber{75}{f2b5}
\createlinenumber{76}{e0a5}
\createlinenumber{77}{59b9}
\createlinenumber{78}{54ff}
\createlinenumber{79}{09f7}
\createlinenumber{80}{acb9}
\createlinenumber{81}{b93d}
\createlinenumber{82}{421c}
\createlinenumber{83}{427e}
\createlinenumber{84}{4aa1}
\createlinenumber{85}{4e9f}
\createlinenumber{86}{7d06}
\createlinenumber{87}{5eea}
\createlinenumber{88}{df79}
\createlinenumber{89}{2a9c}
\createlinenumber{90}{bdae}
\createlinenumber{91}{a09f}
\createlinenumber{92}{95cf}
\createlinenumber{93}{329b}
\createlinenumber{94}{427e}
\createlinenumber{95}{bb73}
\createlinenumber{96}{2bbc}
\createlinenumber{97}{3049}
\createlinenumber{98}{edec}
\createlinenumber{99}{2824}
\createlinenumber{100}{79c7}
\createlinenumber{101}{427e}
\createlinenumber{102}{d35b}
\createlinenumber{103}{5985}
\createlinenumber{104}{d804}
\createlinenumber{105}{5d47}
\createlinenumber{106}{3f3a}
\createlinenumber{107}{0117}
\createlinenumber{108}{cb76}
\createlinenumber{109}{e965}
\createlinenumber{110}{95cf}
\createlinenumber{111}{427e}
\createlinenumber{112}{1e3a}
\createlinenumber{113}{99eb}
\createlinenumber{114}{e949}
\createlinenumber{115}{5a51}
\createlinenumber{116}{95cf}
\createlinenumber{117}{84ad}
\createlinenumber{118}{21d4}
\createlinenumber{119}{81fb}
\createlinenumber{120}{5a4d}
\createlinenumber{121}{5c56}
\createlinenumber{122}{21fc}
\createlinenumber{123}{3d4b}
\createlinenumber{124}{0a05}
\createlinenumber{125}{95cf}
\createlinenumber{126}{95cf}
\createlinenumber{127}{84fb}
\createlinenumber{128}{95cf}
\createlinenumber{129}{f8b9}
\createlinenumber{130}{c62d}
\createlinenumber{131}{a98a}
\createlinenumber{132}{2e8b}
\createlinenumber{133}{2591}
\createlinenumber{134}{79a8}
\createlinenumber{135}{f7cd}
\createlinenumber{136}{6593}
\createlinenumber{137}{a98a}
\createlinenumber{138}{488b}
\createlinenumber{139}{3361}
\createlinenumber{140}{95cf}
\createlinenumber{141}{e54d}
\createlinenumber{142}{2d49}
\createlinenumber{143}{a053}
\createlinenumber{144}{6210}
\createlinenumber{145}{3217}
\createlinenumber{146}{2602}
\createlinenumber{147}{87f6}
\createlinenumber{148}{6953}
\createlinenumber{149}{3c69}
\createlinenumber{150}{4c72}
\createlinenumber{151}{794c}
\createlinenumber{152}{07ec}
\createlinenumber{153}{c7e4}
\createlinenumber{154}{52c5}
\createlinenumber{155}{a4dc}
\createlinenumber{156}{290e}
\createlinenumber{157}{95cf}
\createlinenumber{158}{95cf}
\createlinenumber{159}{95cf}
\createlinenumber{160}{95cf}
\createlinenumber{161}{e8cd}
\createlinenumber{162}{a85c}
\createlinenumber{163}{b4d8}
\createlinenumber{164}{95cf}
\createlinenumber{165}{3f4c}
\createlinenumber{166}{426c}
\createlinenumber{167}{00f0}
\createlinenumber{168}{7fd2}
\createlinenumber{169}{89ad}
\createlinenumber{170}{d921}
\createlinenumber{171}{95cf}
\createlinenumber{172}{95cf}
\createlinenumber{173}{7639}
\createlinenumber{174}{81fc}
\createlinenumber{175}{a69f}
\createlinenumber{176}{a69f}
\createlinenumber{177}{ef2b}
\createlinenumber{178}{42c9}
\createlinenumber{179}{433d}
\createlinenumber{180}{3076}
\createlinenumber{181}{95cf}
\createlinenumber{182}{329b}
\createlinenumber{183}{427e}
\begin{lstlisting}[language=C++]
/* 支持上下界的最小费用S-T可行流
 * 可兼作高效最小费用最大流实现
 * template
 *   <class W>     - the weight type
 * struct MinCostFlow
 *
 * G.init(         - init graph
 *    source node id, 
 *    sink node id, 
 *    node count,  - number of nodes, 0-based
 *    epsilon,     - 0 for integer, 1e-6 for floats
 *    inf,         - the weight upper bound
 * )
 * G.insert(       - insert an edge
 *    u,           - first vertex
 *    v,           - second vertex
 *    lb,          - flow lower bound
 *    ub,          - flow upper bound
 *    cost         - unit edge cost (type is W)
 * )
 * G.work() returns the minimum cost.
 */
#include<iostream>
#include<cstdio>
#include<algorithm>
#include<vector>
#include<queue>
#include<complex>
using namespace std;

template <class W>
struct MinCostFlow {
	struct Edge {
		int u, v, cap, nxt;
		W cost;
		Edge &set(int _u,int _v,int _nxt,int _cap,W _cost) {
			u=_u;v=_v;cost=_cost;cap=_cap;nxt=_nxt;
			return *this;
		}
	};

	vector<Edge> e;
	vector<int> vst, head, que;
	vector<W> dist;
	vector<bool> mark;
	int st, en, N, tot, ret;
	W val, INF, EPS;

	void init(int _s, int _t, int _n, W eps, W inf) {
		head.resize(_n);
		vst.resize(_n);
		dist.resize(_n);
		que.resize(_n);
		mark.resize(_n);
		st=_s;en=_t;N=_n;EPS=eps;INF=inf;
		ret=0;val=0;tot=0;fill(head.begin(),head.end(),-1);
	}

	int dfs(int v, int cap) {
		if(v == en) {
			val += cap * dist[st];
			return cap;
		}
		vst[v]=true;
		int flow=0;
		for(int i=head[v];i!=-1;i=e[i].nxt)
			if(e[i].cap && ! vst[e[i].v] && abs(dist[e[i].v]+e[i].cost-dist[v])<=EPS) {
				int det = dfs(e[i].v, min(cap, e[i].cap));
				if(det) {
					e[i].cap -= det; e[i^1].cap += det; flow += det;
					if(!(cap-=det))break;
				}
			}
		return flow;
	}
	int relabel() {
		W det = INF;
		for(int i=0;i<N;++i)
			if(vst[i])
				for(int j=head[i];j!=-1;j=e[j].nxt)
					if(e[j].cap && !vst[e[j].v])
						det = min(det, dist[e[j].v]+e[j].cost-dist[i]);
		if(det==INF) return false;
		for(int i=0;i<N;++i)
			if(vst[i])dist[i]+=det;
		return true;
	}
	int spfa() {
		queue<int> que;
		fill(vst.begin(),vst.end(),-1);
		vst[en] = -2;
		fill(dist.begin(),dist.end(),INF); dist[en]=0;
		que.push(en);
		fill(mark.begin(),mark.end(),false);
		while(!que.empty()) {
			int u = que.front(); que.pop();
			mark[u]=false;
			for(int i=head[u];i!=-1;i=e[i].nxt)
				if(e[i^1].cap && dist[u]+e[i^1].cost<dist[e[i].v]) {
					dist[e[i].v]=dist[u]+e[i^1].cost;vst[e[i].v]=i^1;
					if(!mark[e[i].v]) {
						mark[e[i].v]=true;
						que.push(e[i].v);
					}
				}
		}
	}
	void insert(int u, int v,int cap, W cost) {
		e.push_back(Edge().set(u,v,head[u],cap,cost)); head[u]=tot++;
		e.push_back(Edge().set(v,u,head[v],0,-cost)); head[v]=tot++;
	}
	void insert(int u, int v, int lo, int hi, W cost) {
		if(hi > lo)
			insert(u, v, hi-lo, cost);
		if(lo > 0) {
			insert(st, v, lo, cost);
			insert(u, en, lo, 0);
		}
	}
	W work() {
		spfa();
		do {
			do {
				fill(vst.begin(),vst.end(),false);
			} while(dfs(st, 2147483647));
		} while(relabel());
		return val;
	}
};

\end{lstlisting}
\subsection{线性规划(单纯形)}
\createlinenumber{184}{427e}
\createlinenumber{185}{427e}
\createlinenumber{186}{427e}
\createlinenumber{187}{427e}
\createlinenumber{188}{4fbb}
\createlinenumber{189}{70b7}
\createlinenumber{190}{053e}
\createlinenumber{191}{3977}
\createlinenumber{192}{f1cf}
\createlinenumber{193}{0938}
\createlinenumber{194}{d471}
\createlinenumber{195}{bdad}
\createlinenumber{196}{f7dc}
\createlinenumber{197}{76b3}
\createlinenumber{198}{e8d0}
\createlinenumber{199}{8cfd}
\createlinenumber{200}{f4db}
\createlinenumber{201}{427e}
\createlinenumber{202}{ce3a}
\createlinenumber{203}{68bc}
\createlinenumber{204}{35b8}
\createlinenumber{205}{d126}
\createlinenumber{206}{dead}
\createlinenumber{207}{427e}
\createlinenumber{208}{83d4}
\createlinenumber{209}{ada3}
\createlinenumber{210}{1e9c}
\createlinenumber{211}{75b7}
\createlinenumber{212}{1080}
\createlinenumber{213}{6814}
\createlinenumber{214}{531b}
\createlinenumber{215}{95cf}
\createlinenumber{216}{6340}
\createlinenumber{217}{1c55}
\createlinenumber{218}{c429}
\createlinenumber{219}{6889}
\createlinenumber{220}{f08e}
\createlinenumber{221}{a898}
\createlinenumber{222}{22bd}
\createlinenumber{223}{5129}
\createlinenumber{224}{f5ca}
\createlinenumber{225}{e2e5}
\createlinenumber{226}{8eb2}
\createlinenumber{227}{95cf}
\createlinenumber{228}{f08e}
\createlinenumber{229}{a898}
\createlinenumber{230}{be8e}
\createlinenumber{231}{55c8}
\createlinenumber{232}{5cdf}
\createlinenumber{233}{f105}
\createlinenumber{234}{8599}
\createlinenumber{235}{0fc0}
\createlinenumber{236}{7bbe}
\createlinenumber{237}{8eb2}
\createlinenumber{238}{95cf}
\createlinenumber{239}{16f0}
\createlinenumber{240}{9394}
\createlinenumber{241}{329f}
\createlinenumber{242}{95cf}
\createlinenumber{243}{427e}
\begin{lstlisting}[language=C++]
// Simplex for linear programming
// min. A(0, 0:n-1)
// s.t. A(1:m, 0:n-1) x <= A(1:m, n)

#define lp for(;;)
#define repf(i,a,b)
#define ft(i,a,b) for (int i=(a);i<=(b);++i)
#define rep(i,n) for (int i=0;i<(n);++i)
#define rtn return
#define pb push_back
#define mp make_pair
#define sz(x) (int((x).size()))
typedef double db;
typedef vector<int> vi;
db inf=1e+10;
db eps=1e-10;
inline int sgn(const db& x){rtn (x>+eps)-(x<-eps);}

const int MAXN=500;
const int MAXM=501;
int n,m;
db A[MAXM+1][MAXN+1],X[MAXN];
int basis[MAXM+1],out[MAXN+1];

void pivot(int a,int b) {
  ft(i,0,m) if (i!=a&&sgn(A[i][b])) ft(j,0,n)
    if (j!=b) A[i][j]-=A[a][j]*A[i][b]/A[a][b];
  ft(j,0,n) if (j!=b) A[a][j]/=A[a][b];
  ft(i,0,m) if (i!=a) A[i][b]/=-A[a][b];
  A[a][b]=1/A[a][b];
  swap(basis[a],out[b]);
}
db simplex() {
  rep(j,n) A[0][j]=-A[0][j];
  ft(i,0,m) basis[i]=-i;
  ft(j,0,n) out[j]=j;
  lp {
    int ii=1,jj=0;
    ft(i,1,m) if (mp(A[i][n],basis[i])<mp(A[ii][n],basis[ii])) ii=i;
    if (A[ii][n]>=0) break;
    rep(j,n) if (A[ii][j]<A[ii][jj]) jj=j;
    if (A[ii][jj]>=0) rtn -inf;
    pivot(ii,jj);
  }
  lp {
    int ii=1,jj=0;
    rep(j,n) if (mp(A[0][j],out[j])<mp(A[0][jj],out[jj])) jj=j;
    if (A[0][jj]>=0) break;
    ft(i,1,m)
      if (A[i][jj]>0&&(A[ii][jj]<=0||mp(A[i][n]/A[i][jj],basis[i])
        <mp(A[ii][n]/A[ii][jj],basis[ii])))
        ii=i;
    if (A[ii][jj]<=0) rtn +inf;
    pivot(ii,jj);
  }
  rep(j,n) X[j]=0;
  ft(i,1,m) if (basis[i]>=0) X[basis[i]]=A[i][n];
  rtn A[0][n];
}

\end{lstlisting}
\section{数论}
\subsection{二次剩余}
\createlinenumber{244}{1915}
\createlinenumber{245}{e0a5}
\createlinenumber{246}{9a97}
\createlinenumber{247}{421c}
\createlinenumber{248}{427e}
\createlinenumber{249}{a7e2}
\createlinenumber{250}{3428}
\createlinenumber{251}{4e00}
\createlinenumber{252}{427e}
\createlinenumber{253}{e79e}
\createlinenumber{254}{025c}
\createlinenumber{255}{95cf}
\createlinenumber{256}{427e}
\createlinenumber{257}{a3a5}
\createlinenumber{258}{0829}
\createlinenumber{259}{4610}
\createlinenumber{260}{4f4f}
\createlinenumber{261}{95cf}
\createlinenumber{262}{427e}
\createlinenumber{263}{f245}
\createlinenumber{264}{251d}
\createlinenumber{265}{8c20}
\createlinenumber{266}{be0f}
\createlinenumber{267}{8fec}
\createlinenumber{268}{4206}
\createlinenumber{269}{95cf}
\createlinenumber{270}{427e}
\createlinenumber{271}{ca53}
\createlinenumber{272}{b045}
\createlinenumber{273}{95cf}
\createlinenumber{274}{427e}
\createlinenumber{275}{3036}
\createlinenumber{276}{6871}
\createlinenumber{277}{bc97}
\createlinenumber{278}{f061}
\createlinenumber{279}{95cf}
\createlinenumber{280}{427e}
\createlinenumber{281}{2ff8}
\createlinenumber{282}{bf85}
\createlinenumber{283}{0a41}
\createlinenumber{284}{6e08}
\createlinenumber{285}{d512}
\createlinenumber{286}{427e}
\createlinenumber{287}{1026}
\createlinenumber{288}{e166}
\createlinenumber{289}{8486}
\createlinenumber{290}{4976}
\createlinenumber{291}{2e24}
\createlinenumber{292}{6645}
\createlinenumber{293}{af97}
\createlinenumber{294}{d074}
\createlinenumber{295}{95cf}
\createlinenumber{296}{95cf}
\createlinenumber{297}{95cf}
\createlinenumber{298}{329b}
\createlinenumber{299}{427e}
\createlinenumber{300}{427e}
\createlinenumber{301}{427e}
\createlinenumber{302}{3117}
\createlinenumber{303}{30aa}
\createlinenumber{304}{427b}
\createlinenumber{305}{8661}
\createlinenumber{306}{3c2f}
\createlinenumber{307}{ac98}
\createlinenumber{308}{44c8}
\createlinenumber{309}{7172}
\createlinenumber{310}{20d0}
\createlinenumber{311}{037f}
\createlinenumber{312}{5354}
\createlinenumber{313}{2b81}
\createlinenumber{314}{95cf}
\createlinenumber{315}{95cf}
\createlinenumber{316}{7021}
\createlinenumber{317}{95cf}
\createlinenumber{318}{427e}
\begin{mdframed}[hidealllines=true,backgroundcolor=blue!5]
求最小的$0\le x < p$满足$x^2=n\ (\mathrm{mod}\ p)$ ($p$是素数)。不存在时返回$-1$。
\end{mdframed}\vspace{-10pt}
\begin{lstlisting}[language=C++]
#include <stdio.h>
#include <iostream>
#include <stdlib.h>
using namespace std;

struct QuadraticResidue {
  typedef long long int64;
  typedef pair<int64, int64> QuadraticNum;

  inline int64 Sqr(int64 a) {
    return a * a;
  }

  int64 Exp(int64 a, int64 n, int64 mod) {
    if (n == 0) return 1;
    if (n & 1) return Sqr(Exp(a, n >> 1, mod)) % mod * a % mod;
    else return Sqr(Exp(a, n >> 1, mod)) % mod;
  }

  int64 inx;
  QuadraticNum QuadraticMul(const QuadraticNum a, const QuadraticNum b, const int mod) {
    QuadraticNum ans;
    ans.first = (a.first * b.first % mod + a.second * b.second % mod * inx % mod) % mod;
    ans.second = (a.first * b.second % mod + a.second * b.first % mod) % mod;
    return ans;
  }

  QuadraticNum Sqr(QuadraticNum a, int mod) {
    return QuadraticMul(a, a, mod);
  }

  QuadraticNum Exp(QuadraticNum a, int n, int mod) {
    if (n == 0) return QuadraticNum(1, 0);
    if (n & 1) return QuadraticMul(Sqr(Exp(a, n >> 1, mod), mod), a, mod);
    else return Sqr(Exp(a, n >> 1, mod), mod);
  }

  int GetRoot(int n, int m) {
    n = (n % m + m) % m;
    if (n == 0) return 0;
    if (m == 2) return 1;
    if ((Exp(n, (m - 1) / 2, m) + m) % m == m - 1) return -1;
    // x does not exist.
    while (true) {
      int i = rand() % m;
      inx = ((int64)i * i - n) % m;
      if ((Exp(inx, (m - 1) / 2, m) + m) % m == m - 1) {
        int x = (Exp(QuadraticNum(i, 1), (m + 1) / 2, m).first + m) % m;
        int y = m - x;
        if (x > y) swap(x, y);
        return x;
      }
    }
  }
};

// This shows how to use it. 
// You can submit on http://acm.timus.ru/problem.aspx?space=1&num=1132
int main() {
  int t, n, m;
  QuadraticResidue quadratic_residue;
  scanf("%d", &t);
  while (t--) {
    scanf("%d%d", &n, &m);
    int root = quadratic_residue.GetRoot(n, m);
    if (root == -1)
      printf("No root\n");
    else {
      if (root + root == m) printf("%d\n", root);
      else printf("%d %d\n", root, m - root);
    }
  }
  return 0;
}

\end{lstlisting}
\section{Watashi代码库(备用)}
\subsection{$O(n\log n)-O(1)$ RMQ}
\createlinenumber{319}{9581}
\createlinenumber{320}{54ff}
\createlinenumber{321}{427e}
\createlinenumber{322}{421c}
\createlinenumber{323}{427e}
\createlinenumber{324}{b7ec}
\createlinenumber{325}{1f3e}
\createlinenumber{326}{5c83}
\createlinenumber{327}{bd3a}
\createlinenumber{328}{f687}
\createlinenumber{329}{427e}
\createlinenumber{330}{de2b}
\createlinenumber{331}{eeb3}
\createlinenumber{332}{449d}
\createlinenumber{333}{427e}
\createlinenumber{334}{6413}
\createlinenumber{335}{cdf9}
\createlinenumber{336}{95cf}
\createlinenumber{337}{427e}
\createlinenumber{338}{01c5}
\createlinenumber{339}{b985}
\createlinenumber{340}{40f2}
\createlinenumber{341}{427e}
\createlinenumber{342}{dcba}
\createlinenumber{343}{f312}
\createlinenumber{344}{1114}
\createlinenumber{345}{95cf}
\createlinenumber{346}{1730}
\createlinenumber{347}{427e}
\createlinenumber{348}{6c2f}
\createlinenumber{349}{ac66}
\createlinenumber{350}{95cf}
\createlinenumber{351}{b0ef}
\createlinenumber{352}{6941}
\createlinenumber{353}{d8a7}
\createlinenumber{354}{95cf}
\createlinenumber{355}{95cf}
\createlinenumber{356}{95cf}
\createlinenumber{357}{427e}
\createlinenumber{358}{7689}
\createlinenumber{359}{083c}
\createlinenumber{360}{db43}
\createlinenumber{361}{95cf}
\createlinenumber{362}{427e}
\createlinenumber{363}{6e14}
\createlinenumber{364}{72b5}
\createlinenumber{365}{95cf}
\createlinenumber{366}{329b}
\createlinenumber{367}{427e}
\createlinenumber{368}{427e}
\begin{lstlisting}[language=C++]
#include <climits>    // CHAR_BIT
#include <algorithm>  // copy

using namespace std;

template<typename T>
struct RMQ {
  int n;
  vector<T> e;
  vector<vector<int> > rmq;

  static const int INT_BIT = sizeof(4) * CHAR_BIT;
  static inline int LG2(int i) { return INT_BIT - 1 - __builtin_clz(i); }
  static inline int BIN(int i) { return 1 << i; }

  int cmp(int l, int r) const {
    return e[l] <= e[r] ? l : r;
  }

  void init(int n, const T e[]) {
    this->n = n;
    vector<T>(e, e + n).swap(this->e);

    int m = 1;
    while (BIN(m) <= n) {
      ++m;
    }
    vector<vector<int> >(m, vector<int>(n)).swap(rmq);

    for (int i = 0; i < n; ++i) {
      rmq[0][i] = i;
    }
    for (int i = 0; BIN(i + 1) <= n; ++i) {
      for (int j = 0; j + BIN(i + 1) <= n; ++j) {
        rmq[i + 1][j] = cmp(rmq[i][j], rmq[i][j + BIN(i)]);
      }
    }
  }

  int index(int l, int r) const {
    int b = LG2(r - l);
    return cmp(rmq[b][l], rmq[b][r - (1 << b)]);
  }

  T value(int l, int r) const {
    return e[index(l, r)];
  }
};


\end{lstlisting}
\subsection{$O(n\log n)-O(\log n)$ LCA}
\createlinenumber{369}{59b9}
\createlinenumber{370}{09f7}
\createlinenumber{371}{54ff}
\createlinenumber{372}{427e}
\createlinenumber{373}{421c}
\createlinenumber{374}{427e}
\createlinenumber{375}{3bd6}
\createlinenumber{376}{b0bb}
\createlinenumber{377}{427e}
\createlinenumber{378}{427e}
\createlinenumber{379}{6dd4}
\createlinenumber{380}{6910}
\createlinenumber{381}{6058}
\createlinenumber{382}{427e}
\createlinenumber{383}{de24}
\createlinenumber{384}{c4a8}
\createlinenumber{385}{67b8}
\createlinenumber{386}{47b3}
\createlinenumber{387}{95cf}
\createlinenumber{388}{6bea}
\createlinenumber{389}{2389}
\createlinenumber{390}{2732}
\createlinenumber{391}{1e13}
\createlinenumber{392}{3245}
\createlinenumber{393}{95cf}
\createlinenumber{394}{95cf}
\createlinenumber{395}{95cf}
\createlinenumber{396}{427e}
\createlinenumber{397}{6074}
\createlinenumber{398}{3b30}
\createlinenumber{399}{1fe0}
\createlinenumber{400}{bbaa}
\createlinenumber{401}{95cf}
\createlinenumber{402}{95cf}
\createlinenumber{403}{aa78}
\createlinenumber{404}{95cf}
\createlinenumber{405}{427e}
\createlinenumber{406}{8119}
\createlinenumber{407}{e7ce}
\createlinenumber{408}{4012}
\createlinenumber{409}{95cf}
\createlinenumber{410}{66a3}
\createlinenumber{411}{59db}
\createlinenumber{412}{5ffd}
\createlinenumber{413}{8e2e}
\createlinenumber{414}{08b5}
\createlinenumber{415}{7ecb}
\createlinenumber{416}{434f}
\createlinenumber{417}{f29b}
\createlinenumber{418}{95cf}
\createlinenumber{419}{95cf}
\createlinenumber{420}{d21f}
\createlinenumber{421}{95cf}
\createlinenumber{422}{95cf}
\createlinenumber{423}{427e}
\createlinenumber{424}{d34f}
\createlinenumber{425}{6c2f}
\createlinenumber{426}{c9ff}
\createlinenumber{427}{95cf}
\createlinenumber{428}{95cf}
\createlinenumber{429}{427e}
\createlinenumber{430}{77e3}
\createlinenumber{431}{486c}
\createlinenumber{432}{800c}
\createlinenumber{433}{95cf}
\createlinenumber{434}{427e}
\createlinenumber{435}{2114}
\createlinenumber{436}{3dbe}
\createlinenumber{437}{e5db}
\createlinenumber{438}{95cf}
\createlinenumber{439}{a83c}
\createlinenumber{440}{427e}
\createlinenumber{441}{427e}
\begin{lstlisting}[language=C++]
#include <cstdio>
#include <vector>
#include <algorithm>

using namespace std;

const int MAXM = 16;
const int MAXN = 1 << MAXM;

// LCA
struct LCA {
  vector<int> e[MAXN];
  int d[MAXN], p[MAXN][MAXM];

  void dfs_(int v, int f) {
    p[v][0] = f;
    for (int i = 1; i < MAXM; ++i) {
      p[v][i] = p[p[v][i - 1]][i - 1];
    }
    for (int i = 0; i < (int)e[v].size(); ++i) {
      int w = e[v][i];
      if (w != f) {
        d[w] = d[v] + 1;
        dfs_(w, v);
      }
    }
  }

  int up_(int v, int m) {
    for (int i = 0; i < MAXM; ++i) {
      if (m & (1 << i)) {
        v = p[v][i];
      }
    }
    return v;
  }

  int lca(int a, int b) {
    if (d[a] > d[b]) {
      swap(a, b);
    }
    b = up_(b, d[b] - d[a]);
    if (a == b) {
      return a;
    } else {
      for (int i = MAXM - 1; i >= 0; --i) {
        if (p[a][i] != p[b][i]) {
          a = p[a][i];
          b = p[b][i];
        }
      }
      return p[a][0];
    }
  }

  void init(int n) {
    for (int i = 0; i < n; ++i) {
      e[i].clear();
    }
  }

  void add(int a, int b) {
    e[a].push_back(b);
    e[b].push_back(a);
  }

  void build() {
    d[0] = 0;
    dfs_(0, 0);
  }
} lca;


\end{lstlisting}
\subsection{树状数组}
\createlinenumber{442}{09f7}
\createlinenumber{443}{427e}
\createlinenumber{444}{421c}
\createlinenumber{445}{427e}
\createlinenumber{446}{ad8a}
\createlinenumber{447}{5f7d}
\createlinenumber{448}{ab5c}
\createlinenumber{449}{427e}
\createlinenumber{450}{d34f}
\createlinenumber{451}{7f29}
\createlinenumber{452}{95cf}
\createlinenumber{453}{427e}
\createlinenumber{454}{853b}
\createlinenumber{455}{13ab}
\createlinenumber{456}{a45a}
\createlinenumber{457}{95cf}
\createlinenumber{458}{95cf}
\createlinenumber{459}{427e}
\createlinenumber{460}{427e}
\createlinenumber{461}{9992}
\createlinenumber{462}{951c}
\createlinenumber{463}{b76f}
\createlinenumber{464}{2a23}
\createlinenumber{465}{95cf}
\createlinenumber{466}{ee0f}
\createlinenumber{467}{95cf}
\createlinenumber{468}{427e}
\createlinenumber{469}{7517}
\createlinenumber{470}{da02}
\createlinenumber{471}{95cf}
\createlinenumber{472}{427e}
\createlinenumber{473}{bb52}
\createlinenumber{474}{7c32}
\createlinenumber{475}{95cf}
\createlinenumber{476}{329b}
\createlinenumber{477}{427e}
\createlinenumber{478}{427e}
\begin{lstlisting}[language=C++]
#include <vector>

using namespace std;

template<typename T = int>
struct BIT {
  vector<T> a;

  void init(int n) {
    vector<T>(n + 1).swap(a);
  }

  void add(int i, T v) {
    for (int j = i + 1; j < (int)a.size(); j = (j | (j - 1)) + 1) {
      a[j] += v;
    }
  }

  // [0, i)
  T sum(int i) const {
    T ret = T();
    for (int j = i; j > 0; j = j & (j - 1)) {
      ret += a[j];
    }
    return ret;
  }

  T get(int i) const {
    return sum(i + 1) - sum(i);
  }

  void set(int i, T v) {
    add(i, v - get(i));
  }
};


\end{lstlisting}
\subsection{并查集}
\createlinenumber{479}{09f7}
\createlinenumber{480}{427e}
\createlinenumber{481}{421c}
\createlinenumber{482}{427e}
\createlinenumber{483}{c793}
\createlinenumber{484}{7521}
\createlinenumber{485}{427e}
\createlinenumber{486}{d34f}
\createlinenumber{487}{9156}
\createlinenumber{488}{6c2f}
\createlinenumber{489}{a208}
\createlinenumber{490}{95cf}
\createlinenumber{491}{95cf}
\createlinenumber{492}{427e}
\createlinenumber{493}{8b08}
\createlinenumber{494}{a584}
\createlinenumber{495}{95cf}
\createlinenumber{496}{427e}
\createlinenumber{497}{7fac}
\createlinenumber{498}{8b35}
\createlinenumber{499}{e8c6}
\createlinenumber{500}{df09}
\createlinenumber{501}{29ee}
\createlinenumber{502}{95cf}
\createlinenumber{503}{329b}
\createlinenumber{504}{427e}
\begin{lstlisting}[language=C++]
#include <vector>

using namespace std;

struct DisjointSet {
  vector<int> p;

  void init(int n) {
    p.resize(n);
    for (int i = 0; i < n; ++i) {
      p[i] = i;
    }
  }

  int getp(int i) {
    return i == p[i] ? i : (p[i] = getp(p[i]));
  }

  bool setp(int i, int j) {
    i = getp(i);
    j = getp(j);
    p[i] = j;
    return i != j;
  }
};

\end{lstlisting}
\subsection{轻重权树剖分}
\createlinenumber{505}{59b9}
\createlinenumber{506}{09f7}
\createlinenumber{507}{54ff}
\createlinenumber{508}{427e}
\createlinenumber{509}{421c}
\createlinenumber{510}{427e}
\createlinenumber{511}{3bd6}
\createlinenumber{512}{b0bb}
\createlinenumber{513}{427e}
\createlinenumber{514}{427e}
\createlinenumber{515}{b49e}
\createlinenumber{516}{1769}
\createlinenumber{517}{4971}
\createlinenumber{518}{9524}
\createlinenumber{519}{448c}
\createlinenumber{520}{a6ab}
\createlinenumber{521}{d16b}
\createlinenumber{522}{427e}
\createlinenumber{523}{de24}
\createlinenumber{524}{e350}
\createlinenumber{525}{5172}
\createlinenumber{526}{6bea}
\createlinenumber{527}{2389}
\createlinenumber{528}{2732}
\createlinenumber{529}{3245}
\createlinenumber{530}{ea7a}
\createlinenumber{531}{95cf}
\createlinenumber{532}{95cf}
\createlinenumber{533}{95cf}
\createlinenumber{534}{427e}
\createlinenumber{535}{a22f}
\createlinenumber{536}{fd9c}
\createlinenumber{537}{048a}
\createlinenumber{538}{eb44}
\createlinenumber{539}{427e}
\createlinenumber{540}{e6d1}
\createlinenumber{541}{6bea}
\createlinenumber{542}{2389}
\createlinenumber{543}{2732}
\createlinenumber{544}{08e1}
\createlinenumber{545}{3089}
\createlinenumber{546}{35ea}
\createlinenumber{547}{95cf}
\createlinenumber{548}{95cf}
\createlinenumber{549}{95cf}
\createlinenumber{550}{bee4}
\createlinenumber{551}{f904}
\createlinenumber{552}{95cf}
\createlinenumber{553}{427e}
\createlinenumber{554}{6bea}
\createlinenumber{555}{2389}
\createlinenumber{556}{6e71}
\createlinenumber{557}{9768}
\createlinenumber{558}{95cf}
\createlinenumber{559}{95cf}
\createlinenumber{560}{95cf}
\createlinenumber{561}{427e}
\createlinenumber{562}{d34f}
\createlinenumber{563}{6c2f}
\createlinenumber{564}{c9ff}
\createlinenumber{565}{95cf}
\createlinenumber{566}{95cf}
\createlinenumber{567}{427e}
\createlinenumber{568}{77e3}
\createlinenumber{569}{486c}
\createlinenumber{570}{800c}
\createlinenumber{571}{95cf}
\createlinenumber{572}{427e}
\createlinenumber{573}{2114}
\createlinenumber{574}{d7ef}
\createlinenumber{575}{e5db}
\createlinenumber{576}{917b}
\createlinenumber{577}{95cf}
\createlinenumber{578}{8dca}
\createlinenumber{579}{427e}
\createlinenumber{580}{427e}
\begin{lstlisting}[language=C++]
#include <cstdio>
#include <vector>
#include <algorithm>

using namespace std;

const int MAXM = 16;
const int MAXN = 1 << MAXM;

// Heavy-Light Decomposition
struct TreeDecomposition {
  vector<int> e[MAXN], c[MAXN];
  int s[MAXN];    // subtree size
  int p[MAXN];    // parent id
  int r[MAXN];    // chain root id
  int t[MAXN];    // timestamp, index used in segtree
  int ts;

  void dfs_(int v, int f) {
    p[v] = f;
    s[v] = 1;
    for (int i = 0; i < (int)e[v].size(); ++i) {
      int w = e[v][i];
      if (w != f) {
        dfs_(w, v);
        s[v] += s[w];
      }
    }
  }

  void decomp_(int v, int f, int k) {
    t[v] = ts++;
    c[k].push_back(v);
    r[v] = k;

    int x = 0, y = -1;
    for (int i = 0; i < (int)e[v].size(); ++i) {
      int w = e[v][i];
      if (w != f) {
        if (s[w] > x) {
          x = s[w];
          y = w;
        }
      }
    }
    if (y != -1) {
      decomp_(y, v, k);
    }

    for (int i = 0; i < (int)e[v].size(); ++i) {
      int w = e[v][i];
      if (w != f && w != y) {
        decomp_(w, v, w);
      }
    }
  }

  void init(int n) {
    for (int i = 0; i < n; ++i) {
      e[i].clear();
    }
  }

  void add(int a, int b) {
    e[a].push_back(b);
    e[b].push_back(a);
  }

  void build() {  // !!
    ts = 0;
    dfs_(0, 0);
    decomp_(0, 0, 0);
  }
} hld;


\end{lstlisting}
\subsection{强连通分量}
\createlinenumber{581}{8207}
\createlinenumber{582}{09f7}
\createlinenumber{583}{54ff}
\createlinenumber{584}{427e}
\createlinenumber{585}{421c}
\createlinenumber{586}{427e}
\createlinenumber{587}{9356}
\createlinenumber{588}{5c83}
\createlinenumber{589}{9d4c}
\createlinenumber{590}{427e}
\createlinenumber{591}{fbd2}
\createlinenumber{592}{e01d}
\createlinenumber{593}{427e}
\createlinenumber{594}{d34f}
\createlinenumber{595}{b985}
\createlinenumber{596}{4883}
\createlinenumber{597}{a1b9}
\createlinenumber{598}{94d6}
\createlinenumber{599}{021f}
\createlinenumber{600}{95cf}
\createlinenumber{601}{427e}
\createlinenumber{602}{77e3}
\createlinenumber{603}{486c}
\createlinenumber{604}{95cf}
\createlinenumber{605}{427e}
\createlinenumber{606}{5728}
\createlinenumber{607}{724e}
\createlinenumber{608}{9cad}
\createlinenumber{609}{427e}
\createlinenumber{610}{3dd3}
\createlinenumber{611}{d2b0}
\createlinenumber{612}{daec}
\createlinenumber{613}{d819}
\createlinenumber{614}{b9fe}
\createlinenumber{615}{650e}
\createlinenumber{616}{0402}
\createlinenumber{617}{181c}
\createlinenumber{618}{e5cc}
\createlinenumber{619}{f480}
\createlinenumber{620}{95cf}
\createlinenumber{621}{95cf}
\createlinenumber{622}{427e}
\createlinenumber{623}{01f5}
\createlinenumber{624}{8631}
\createlinenumber{625}{a69f}
\createlinenumber{626}{5973}
\createlinenumber{627}{c2f4}
\createlinenumber{628}{11f8}
\createlinenumber{629}{7783}
\createlinenumber{630}{b819}
\createlinenumber{631}{e8fd}
\createlinenumber{632}{6cf2}
\createlinenumber{633}{95cf}
\createlinenumber{634}{95cf}
\createlinenumber{635}{427e}
\createlinenumber{636}{3cca}
\createlinenumber{637}{efaa}
\createlinenumber{638}{4810}
\createlinenumber{639}{22e0}
\createlinenumber{640}{427e}
\createlinenumber{641}{d38c}
\createlinenumber{642}{6c2f}
\createlinenumber{643}{e5ea}
\createlinenumber{644}{4ee6}
\createlinenumber{645}{95cf}
\createlinenumber{646}{95cf}
\createlinenumber{647}{d58b}
\createlinenumber{648}{95cf}
\createlinenumber{649}{329b}
\createlinenumber{650}{427e}
\createlinenumber{651}{427e}
\begin{lstlisting}[language=C++]
#include <stack>
#include <vector>
#include <algorithm>

using namespace std;

struct SCCTarjan {
  int n;
  vector<vector<int> > e;

  vector<int> id;
  vector<vector<int> > scc;

  void init(int n) {
    this->n = n;
    vector<vector<int> >(n).swap(e);
    id.resize(n);
    dfn.resize(n);
    low.resize(n);
  }

  void add(int a, int b) {
    e[a].push_back(b);
  }

  vector<int> dfn, low;
  int timestamp;
  stack<int> s;

  void dfs(int v) {
    dfn[v] = timestamp++;
    low[v] = dfn[v];
    s.push(v);
    for (vector<int>::const_iterator w = e[v].begin(); w != e[v].end(); ++w) {
      if (dfn[*w] == -1) {
        dfs(*w);
        low[v] = min(low[v], low[*w]);
      } else if (dfn[*w] != -2) {
        low[v] = min(low[v], dfn[*w]);
      }
    }

    if (low[v] == dfn[v]) {
      vector<int> t;
      do {
        int w = s.top();
        s.pop();
        id[w] = (int)scc.size();
        t.push_back(w);
        dfn[w] = -2;
      } while (t.back() != v);
      scc.push_back(t);
    }
  }

  int gao() {
    scc.clear();
    stack<int>().swap(s);
    timestamp = 0;

    fill(dfn.begin(), dfn.end(), -1);
    for (int i = 0; i < n; ++i) {
      if (dfn[i] == -1) {
        dfs(i);
      }
    }
    return (int)scc.size();
  }
};


\end{lstlisting}
\subsection{双连通分量}
\createlinenumber{652}{8207}
\createlinenumber{653}{09f7}
\createlinenumber{654}{0947}
\createlinenumber{655}{54ff}
\createlinenumber{656}{427e}
\createlinenumber{657}{421c}
\createlinenumber{658}{427e}
\createlinenumber{659}{427e}
\createlinenumber{660}{0ff2}
\createlinenumber{661}{5c83}
\createlinenumber{662}{9d4c}
\createlinenumber{663}{427e}
\createlinenumber{664}{b973}
\createlinenumber{665}{5b05}
\createlinenumber{666}{2eab}
\createlinenumber{667}{427e}
\createlinenumber{668}{d34f}
\createlinenumber{669}{b985}
\createlinenumber{670}{c568}
\createlinenumber{671}{bea5}
\createlinenumber{672}{94d6}
\createlinenumber{673}{021f}
\createlinenumber{674}{95cf}
\createlinenumber{675}{427e}
\createlinenumber{676}{77e3}
\createlinenumber{677}{427e}
\createlinenumber{678}{486c}
\createlinenumber{679}{800c}
\createlinenumber{680}{95cf}
\createlinenumber{681}{427e}
\createlinenumber{682}{5728}
\createlinenumber{683}{724e}
\createlinenumber{684}{76f7}
\createlinenumber{685}{427e}
\createlinenumber{686}{98cc}
\createlinenumber{687}{bb18}
\createlinenumber{688}{9db1}
\createlinenumber{689}{b9fe}
\createlinenumber{690}{05c2}
\createlinenumber{691}{650e}
\createlinenumber{692}{7328}
\createlinenumber{693}{9bb2}
\createlinenumber{694}{181c}
\createlinenumber{695}{c27b}
\createlinenumber{696}{427e}
\createlinenumber{697}{235d}
\createlinenumber{698}{5659}
\createlinenumber{699}{95cf}
\createlinenumber{700}{427e}
\createlinenumber{701}{def0}
\createlinenumber{702}{fc5c}
\createlinenumber{703}{95cf}
\createlinenumber{704}{427e}
\createlinenumber{705}{31f9}
\createlinenumber{706}{a69f}
\createlinenumber{707}{d8ed}
\createlinenumber{708}{c2f4}
\createlinenumber{709}{0e21}
\createlinenumber{710}{79d2}
\createlinenumber{711}{95cf}
\createlinenumber{712}{3597}
\createlinenumber{713}{7328}
\createlinenumber{714}{f480}
\createlinenumber{715}{95cf}
\createlinenumber{716}{95cf}
\createlinenumber{717}{95cf}
\createlinenumber{718}{427e}
\createlinenumber{719}{eb55}
\createlinenumber{720}{052d}
\createlinenumber{721}{7a64}
\createlinenumber{722}{2b46}
\createlinenumber{723}{427e}
\createlinenumber{724}{22e0}
\createlinenumber{725}{ef66}
\createlinenumber{726}{d38c}
\createlinenumber{727}{427e}
\createlinenumber{728}{6c2f}
\createlinenumber{729}{e5ea}
\createlinenumber{730}{e343}
\createlinenumber{731}{95cf}
\createlinenumber{732}{95cf}
\createlinenumber{733}{95cf}
\createlinenumber{734}{329b}
\createlinenumber{735}{427e}
\createlinenumber{736}{427e}
\createlinenumber{737}{0f18}
\createlinenumber{738}{43ad}
\createlinenumber{739}{76c3}
\createlinenumber{740}{6910}
\createlinenumber{741}{427e}
\createlinenumber{742}{d34f}
\createlinenumber{743}{5e72}
\createlinenumber{744}{37f8}
\createlinenumber{745}{95cf}
\createlinenumber{746}{427e}
\createlinenumber{747}{77e3}
\createlinenumber{748}{1cf2}
\createlinenumber{749}{95cf}
\createlinenumber{750}{427e}
\createlinenumber{751}{eb55}
\createlinenumber{752}{e6ec}
\createlinenumber{753}{87a1}
\createlinenumber{754}{75dc}
\createlinenumber{755}{3273}
\createlinenumber{756}{e387}
\createlinenumber{757}{95cf}
\createlinenumber{758}{95cf}
\createlinenumber{759}{95cf}
\createlinenumber{760}{cdda}
\createlinenumber{761}{28e1}
\createlinenumber{762}{9ba6}
\createlinenumber{763}{486c}
\createlinenumber{764}{800c}
\createlinenumber{765}{95cf}
\createlinenumber{766}{95cf}
\createlinenumber{767}{427e}
\createlinenumber{768}{c12a}
\createlinenumber{769}{deff}
\createlinenumber{770}{95cf}
\createlinenumber{771}{329b}
\createlinenumber{772}{427e}
\begin{lstlisting}[language=C++]
#include <stack>
#include <vector>
#include <utility>
#include <algorithm>

using namespace std;

// TODO: cannot handle duplicate edges
struct Tarjan {
  int n;
  vector<vector<int> > e;

  vector<int> cut;
  vector<pair<int, int> > bridge;
  vector<vector<pair<int, int> > > bcc;

  void init(int n) {
    this->n = n;
    e.clear();
    e.resize(n);
    dfn.resize(n);
    low.resize(n);
  }

  void add(int a, int b) {
    // assert(find(e[a].begin(), e[a].end(), b) == e[a].end());
    e[a].push_back(b);
    e[b].push_back(a);
  }

  vector<int> dfn, low;
  int timestamp;
  stack<pair<int, int> > s;

  void dfs(int v, int p) {
    int part = p == -1 ? 0 : 1;
    dfn[v] = low[v] = timestamp++;
    for (vector<int>::const_iterator w = e[v].begin(); w != e[v].end(); ++w) {
      pair<int, int> f = make_pair(min(v, *w), max(v, *w));
      if (dfn[*w] == -1) {
        s.push(f);
        dfs(*w, v);
        low[v] = min(low[v], low[*w]);
        if (dfn[v] <= low[*w]) {
          // articulation point
          if (++part == 2) {
            cut.push_back(v);
          }
          // articulation edge
          if (dfn[v] < low[*w]) {
            bridge.push_back(f);
          }
          // biconnected component (2-vertex-connected)
          vector<pair<int, int> > t;
          do {
            t.push_back(s.top());
            s.pop();
          } while (t.back() != f);
          bcc.push_back(t);
        }
      } else if (*w != p && dfn[*w] < dfn[v]) {
        s.push(f);
        low[v] = min(low[v], dfn[*w]);
      }
    }
  }

  void gao() {
    cut.clear();
    bridge.clear();
    bcc.clear();

    timestamp = 0;
    stack<pair<int, int> >().swap(s);
    fill(dfn.begin(), dfn.end(), -1);

    for (int i = 0; i < n; ++i) {
      if (dfn[i] == -1) {
        dfs(i, -1);
      }
    }
  }
};


struct BridgeBlockTree {
  Tarjan<MAXN> bcc;
  DisjointSet<MAXN> ds;
  vector<int> e[MAXN];

  void init(int n) {
    bcc.init(n);
    ds.init(n);
  }

  void add(int a, int b) {
    bcc.add(a, b);
  }

  void gao() {
    bcc.gao();
    for (const auto& i: bcc.bcc) {
      if (i.size() > 1) {
        for (const auto& j: i) {
          ds.setp(j.first, j.second);
        }
      }
    }
    for (const auto& i: bcc.bridge) {
      int a = ds.getp(i.first);
      int b = ds.getp(i.second);
      e[a].push_back(b);
      e[b].push_back(a);
    }
  }

  int id(int v) {
    return ds.getp(v);
  }
};

\end{lstlisting}
\subsection{二分图匹配}
\createlinenumber{773}{427e}
\createlinenumber{774}{427e}
\createlinenumber{775}{427e}
\createlinenumber{776}{427e}
\createlinenumber{777}{09f7}
\createlinenumber{778}{2349}
\createlinenumber{779}{54ff}
\createlinenumber{780}{427e}
\createlinenumber{781}{421c}
\createlinenumber{782}{427e}
\createlinenumber{783}{84ee}
\createlinenumber{784}{fbf6}
\createlinenumber{785}{9ec6}
\createlinenumber{786}{9d4c}
\createlinenumber{787}{427e}
\createlinenumber{788}{8324}
\createlinenumber{789}{c1d1}
\createlinenumber{790}{f9c1}
\createlinenumber{791}{8789}
\createlinenumber{792}{50d0}
\createlinenumber{793}{c568}
\createlinenumber{794}{25b9}
\createlinenumber{795}{1023}
\createlinenumber{796}{95cf}
\createlinenumber{797}{427e}
\createlinenumber{798}{77e3}
\createlinenumber{799}{486c}
\createlinenumber{800}{95cf}
\createlinenumber{801}{427e}
\createlinenumber{802}{427e}
\createlinenumber{803}{50e0}
\createlinenumber{804}{427e}
\createlinenumber{805}{0c2b}
\createlinenumber{806}{207c}
\createlinenumber{807}{dae4}
\createlinenumber{808}{40c5}
\createlinenumber{809}{4d94}
\createlinenumber{810}{159a}
\createlinenumber{811}{c291}
\createlinenumber{812}{3361}
\createlinenumber{813}{95cf}
\createlinenumber{814}{95cf}
\createlinenumber{815}{95cf}
\createlinenumber{816}{438e}
\createlinenumber{817}{95cf}
\createlinenumber{818}{427e}
\createlinenumber{819}{3cca}
\createlinenumber{820}{5b57}
\createlinenumber{821}{103b}
\createlinenumber{822}{319d}
\createlinenumber{823}{d0cb}
\createlinenumber{824}{87f6}
\createlinenumber{825}{7c0c}
\createlinenumber{826}{8bb8}
\createlinenumber{827}{95cf}
\createlinenumber{828}{95cf}
\createlinenumber{829}{ee0f}
\createlinenumber{830}{95cf}
\createlinenumber{831}{329b}
\createlinenumber{832}{427e}
\createlinenumber{833}{427e}
\begin{lstlisting}[language=C++]
// maximum matchings in bipartite graphs
// maximum cardinality bipartite matching
// O(|V||E|), generally fast

#include <vector>
#include <string>
#include <algorithm>

using namespace std;

struct Hungarian {
  int nx, ny;
  vector<int> mx, my;
  vector<vector<int> > e;

  void init(int nx, int ny) {
    this->nx = nx;
    this->ny = ny;
    mx.resize(nx);
    my.resize(ny);
    e.clear();
    e.resize(nx);
    mark.resize(nx);
  }

  void add(int a, int b) {
    e[a].push_back(b);
  }

  // vector<bool> is evil!!!
  basic_string<bool> mark;

  bool augment(int i) {
    if (!mark[i]) {
      mark[i] = true;
      for (vector<int>::const_iterator j = e[i].begin(); j != e[i].end(); ++j) {
        if (my[*j] == -1 || augment(my[*j])) {
          mx[i] = *j;
          my[*j] = i;
          return true;
        }
      }
    }
    return false;
  }

  int gao() {
    int ret = 0;
    fill(mx.begin(), mx.end(), -1);
    fill(my.begin(), my.end(), -1);
    for (int i = 0; i < nx; ++i) {
      fill(mark.begin(), mark.end(), false);
      if (augment(i)) {
        ++ret;
      }
    }
    return ret;
  }
};


\end{lstlisting}
\subsection{最小费用最大流}
\createlinenumber{834}{acb9}
\createlinenumber{835}{3160}
\createlinenumber{836}{09f7}
\createlinenumber{837}{59b9}
\createlinenumber{838}{54ff}
\createlinenumber{839}{427e}
\createlinenumber{840}{421c}
\createlinenumber{841}{427e}
\createlinenumber{842}{9889}
\createlinenumber{843}{e90e}
\createlinenumber{844}{7b7f}
\createlinenumber{845}{329b}
\createlinenumber{846}{427e}
\createlinenumber{847}{7d06}
\createlinenumber{848}{3b67}
\createlinenumber{849}{4f94}
\createlinenumber{850}{3345}
\createlinenumber{851}{5b0e}
\createlinenumber{852}{7f53}
\createlinenumber{853}{329b}
\createlinenumber{854}{427e}
\createlinenumber{855}{b80f}
\createlinenumber{856}{360a}
\createlinenumber{857}{427e}
\createlinenumber{858}{41c4}
\createlinenumber{859}{b985}
\createlinenumber{860}{a099}
\createlinenumber{861}{8a8f}
\createlinenumber{862}{6c2f}
\createlinenumber{863}{c9ff}
\createlinenumber{864}{95cf}
\createlinenumber{865}{95cf}
\createlinenumber{866}{427e}
\createlinenumber{867}{46f2}
\createlinenumber{868}{7f83}
\createlinenumber{869}{0c8e}
\createlinenumber{870}{95cf}
\createlinenumber{871}{427e}
\createlinenumber{872}{cd81}
\createlinenumber{873}{d6ad}
\createlinenumber{874}{97c2}
\createlinenumber{875}{5e7d}
\createlinenumber{876}{e4d7}
\createlinenumber{877}{427e}
\createlinenumber{878}{70a8}
\createlinenumber{879}{93d2}
\createlinenumber{880}{fef6}
\createlinenumber{881}{eb72}
\createlinenumber{882}{b992}
\createlinenumber{883}{d6ae}
\createlinenumber{884}{5d6d}
\createlinenumber{885}{ce15}
\createlinenumber{886}{ef7c}
\createlinenumber{887}{5ee3}
\createlinenumber{888}{427e}
\createlinenumber{889}{bd63}
\createlinenumber{890}{cc78}
\createlinenumber{891}{e23b}
\createlinenumber{892}{ad9c}
\createlinenumber{893}{15dd}
\createlinenumber{894}{45d3}
\createlinenumber{895}{c425}
\createlinenumber{896}{fd54}
\createlinenumber{897}{b333}
\createlinenumber{898}{95cf}
\createlinenumber{899}{427e}
\createlinenumber{900}{757c}
\createlinenumber{901}{227a}
\createlinenumber{902}{2767}
\createlinenumber{903}{e8de}
\createlinenumber{904}{dbc3}
\createlinenumber{905}{a779}
\createlinenumber{906}{16f9}
\createlinenumber{907}{438e}
\createlinenumber{908}{95cf}
\createlinenumber{909}{05f0}
\createlinenumber{910}{8c3b}
\createlinenumber{911}{065b}
\createlinenumber{912}{eab7}
\createlinenumber{913}{95cf}
\createlinenumber{914}{95cf}
\createlinenumber{915}{95cf}
\createlinenumber{916}{95cf}
\createlinenumber{917}{3361}
\createlinenumber{918}{95cf}
\createlinenumber{919}{427e}
\createlinenumber{920}{93fe}
\createlinenumber{921}{29a5}
\createlinenumber{922}{9e20}
\createlinenumber{923}{1026}
\createlinenumber{924}{9653}
\createlinenumber{925}{c891}
\createlinenumber{926}{8767}
\createlinenumber{927}{6173}
\createlinenumber{928}{8e2e}
\createlinenumber{929}{3ee3}
\createlinenumber{930}{109e}
\createlinenumber{931}{27e0}
\createlinenumber{932}{427e}
\createlinenumber{933}{6fb2}
\createlinenumber{934}{a9d1}
\createlinenumber{935}{8a2b}
\createlinenumber{936}{0cc2}
\createlinenumber{937}{aad1}
\createlinenumber{938}{3d0a}
\createlinenumber{939}{84d7}
\createlinenumber{940}{95cf}
\createlinenumber{941}{95cf}
\createlinenumber{942}{95cf}
\createlinenumber{943}{d460}
\createlinenumber{944}{95cf}
\createlinenumber{945}{329b}
\createlinenumber{946}{427e}
\createlinenumber{947}{427e}
\begin{lstlisting}[language=C++]
#include <queue>
#include <limits>
#include <vector>
#include <cstdio>
#include <algorithm>

using namespace std;

template<int MAXN, typename T = int, typename S = T>
struct MinCostMaxFlow {
  struct NegativeCostCircuitExistsException {
  };

  struct Edge {
    int v;
    T c;
    S w;
    int b;
    Edge(int v, T c, S w, int b) : v(v), c(c), w(w), b(b) { }
  };

  int n, source, sink;
  vector<Edge> e[MAXN];

  void init(int n, int source, int sink) {
    this->n = n;
    this->source = source;
    this->sink = sink;
    for (int i = 0; i < n; ++i) {
      e[i].clear();
    }
  }

  void addEdge(int a, int b, T c, S w) {
    e[a].push_back(Edge(b, c, w, e[b].size()));
    e[b].push_back(Edge(a, 0, -w, e[a].size() - 1));  // TODO
  }

  bool mark[MAXN];
  T maxc[MAXN];
  S minw[MAXN];
  int dist[MAXN];
  Edge* prev[MAXN];

  bool _spfa() {
    queue<int> q;
    fill(mark, mark + n, false);
    fill(maxc, maxc + n, 0);
    fill(minw, minw + n, numeric_limits<S>::max());
    fill(dist, dist + n, 0);
    fill(prev, prev + n, (Edge*)NULL);
    mark[source] = true;
    maxc[source] = numeric_limits<S>::max();
    minw[source] = 0;

    q.push(source);
    while (!q.empty()) {
      int cur = q.front();
      mark[cur] = false;
      q.pop();
      for (typename vector<Edge>::iterator it = e[cur].begin(); it != e[cur].end(); ++it) {
        T c = min(maxc[cur], it->c);
        if (c == 0) {
          continue;
        }

        int v = it->v;
        S w = minw[cur] + it->w;
        if (minw[v] > w || (minw[v] == w && maxc[v] < c)) { // TODO
          maxc[v] = c;
          minw[v] = w;
          dist[v] = dist[cur] + 1;
          if (dist[v] >= n) {
            return false;
          }
          prev[v] = &*it;
          if (!mark[v]) {
            mark[v] = true;
            q.push(v);
          }
        }
      }
    }
    return true;
  }

  pair<T, S> gao() {
    T sumc = 0;
    S sumw = 0;
    while (true) {
      if (!_spfa()) {
        throw NegativeCostCircuitExistsException();
      } else if (maxc[sink] == 0) {
        break;
      } else {
        T c = maxc[sink];
        sumc += c;
        sumw += c * minw[sink];

        int cur = sink;
        while (cur != source) {
          Edge* e1 = prev[cur];
          e1->c -= c;
          Edge* e2 = &e[e1->v][e1->b];
          e2->c += c;
          cur = e2->v;
        }
      }
    }
    return make_pair(sumc, sumw);
  }
};


\end{lstlisting}
\subsection{AhoCorasick自动机}
\createlinenumber{948}{acb9}
\createlinenumber{949}{54ff}
\createlinenumber{950}{427e}
\createlinenumber{951}{421c}
\createlinenumber{952}{427e}
\createlinenumber{953}{3de7}
\createlinenumber{954}{36aa}
\createlinenumber{955}{35e5}
\createlinenumber{956}{2eee}
\createlinenumber{957}{427e}
\createlinenumber{958}{4022}
\createlinenumber{959}{35a5}
\createlinenumber{960}{9143}
\createlinenumber{961}{f098}
\createlinenumber{962}{427e}
\createlinenumber{963}{5d53}
\createlinenumber{964}{84fb}
\createlinenumber{965}{572c}
\createlinenumber{966}{feb8}
\createlinenumber{967}{95cf}
\createlinenumber{968}{427e}
\createlinenumber{969}{a3fb}
\createlinenumber{970}{539d}
\createlinenumber{971}{5b80}
\createlinenumber{972}{e72e}
\createlinenumber{973}{6656}
\createlinenumber{974}{0081}
\createlinenumber{975}{bb4e}
\createlinenumber{976}{95cf}
\createlinenumber{977}{a009}
\createlinenumber{978}{47f8}
\createlinenumber{979}{95cf}
\createlinenumber{980}{e149}
\createlinenumber{981}{95cf}
\createlinenumber{982}{427e}
\createlinenumber{983}{2114}
\createlinenumber{984}{dfc8}
\createlinenumber{985}{a7a6}
\createlinenumber{986}{3873}
\createlinenumber{987}{131c}
\createlinenumber{988}{9b4d}
\createlinenumber{989}{79f5}
\createlinenumber{990}{8e2e}
\createlinenumber{991}{6c43}
\createlinenumber{992}{95cf}
\createlinenumber{993}{95cf}
\createlinenumber{994}{88bb}
\createlinenumber{995}{e42e}
\createlinenumber{996}{38ff}
\createlinenumber{997}{1a76}
\createlinenumber{998}{3873}
\createlinenumber{999}{9f81}
\createlinenumber{1000}{076e}
\createlinenumber{1001}{659d}
\createlinenumber{1002}{8e2e}
\createlinenumber{1003}{7720}
\createlinenumber{1004}{95cf}
\createlinenumber{1005}{95cf}
\createlinenumber{1006}{95cf}
\createlinenumber{1007}{95cf}
\createlinenumber{1008}{0244}
\createlinenumber{1009}{427e}
\createlinenumber{1010}{427e}
\begin{lstlisting}[language=C++]
#include <queue>
#include <algorithm>

using namespace std;

struct AhoCorasick {
  static const int NONE = 0;
  static const int MAXN = 1024;
  static const int CHARSET = 26;

  int end;
  int tag[MAXN];
  int fail[MAXN];
  int trie[MAXN][CHARSET];

  void init() {
    tag[0] = NONE;
    fill(trie[0], trie[0] + CHARSET, -1);
    end = 1;
  }

  int add(int m, const int* s) {
    int p = 0;
    for (int i = 0; i < m; ++i) {
      if (trie[p][*s] == -1) {
        tag[end] = NONE;
        fill(trie[end], trie[end] + CHARSET, -1);
        trie[p][*s] = end++;
      }
      p = trie[p][*s];
      ++s;
    }
    return p;
  }

  void build() {  // !!
    queue<int> bfs;
    fail[0] = 0;
    for (int i = 0; i < CHARSET; ++i) {
      if (trie[0][i] != -1) {
        fail[trie[0][i]] = 0;
        bfs.push(trie[0][i]);
      } else {
        trie[0][i] = 0;
      }
    }
    while (!bfs.empty()) {
      int p = bfs.front();
      tag[p] |= tag[fail[p]];
      bfs.pop();
      for (int i = 0; i < CHARSET; ++i) {
        if (trie[p][i] != -1) {
          fail[trie[p][i]] = trie[fail[p]][i];
          bfs.push(trie[p][i]);
        } else {
          trie[p][i] = trie[fail[p]][i];
        }
      }
    }
  }
} ac;


\end{lstlisting}
\subsection{后缀数组}
\createlinenumber{1011}{09f7}
\createlinenumber{1012}{0947}
\createlinenumber{1013}{54ff}
\createlinenumber{1014}{421c}
\createlinenumber{1015}{427e}
\createlinenumber{1016}{010b}
\createlinenumber{1017}{8a07}
\createlinenumber{1018}{427e}
\createlinenumber{1019}{b7ec}
\createlinenumber{1020}{e220}
\createlinenumber{1021}{b104}
\createlinenumber{1022}{0be0}
\createlinenumber{1023}{427e}
\createlinenumber{1024}{caf4}
\createlinenumber{1025}{6c2f}
\createlinenumber{1026}{7d91}
\createlinenumber{1027}{95cf}
\createlinenumber{1028}{7c59}
\createlinenumber{1029}{6c2f}
\createlinenumber{1030}{079a}
\createlinenumber{1031}{bc80}
\createlinenumber{1032}{ff4f}
\createlinenumber{1033}{8e2e}
\createlinenumber{1034}{dc5c}
\createlinenumber{1035}{95cf}
\createlinenumber{1036}{95cf}
\createlinenumber{1037}{427e}
\createlinenumber{1038}{90d1}
\createlinenumber{1039}{1d87}
\createlinenumber{1040}{9a38}
\createlinenumber{1041}{427e}
\createlinenumber{1042}{5b80}
\createlinenumber{1043}{0ece}
\createlinenumber{1044}{95cf}
\createlinenumber{1045}{5c61}
\createlinenumber{1046}{dcfb}
\createlinenumber{1047}{0041}
\createlinenumber{1048}{95cf}
\createlinenumber{1049}{95cf}
\createlinenumber{1050}{427e}
\createlinenumber{1051}{c9ea}
\createlinenumber{1052}{6c2f}
\createlinenumber{1053}{e3e0}
\createlinenumber{1054}{95cf}
\createlinenumber{1055}{6da6}
\createlinenumber{1056}{1894}
\createlinenumber{1057}{8376}
\createlinenumber{1058}{95cf}
\createlinenumber{1059}{427e}
\createlinenumber{1060}{6c2f}
\createlinenumber{1061}{cc5d}
\createlinenumber{1062}{a959}
\createlinenumber{1063}{95cf}
\createlinenumber{1064}{6c2f}
\createlinenumber{1065}{e1ef}
\createlinenumber{1066}{ff4f}
\createlinenumber{1067}{8e2e}
\createlinenumber{1068}{dc5c}
\createlinenumber{1069}{95cf}
\createlinenumber{1070}{95cf}
\createlinenumber{1071}{95cf}
\createlinenumber{1072}{427e}
\createlinenumber{1073}{3ce6}
\createlinenumber{1074}{ef3c}
\createlinenumber{1075}{d157}
\createlinenumber{1076}{39a0}
\createlinenumber{1077}{8e2e}
\createlinenumber{1078}{691f}
\createlinenumber{1079}{b5d4}
\createlinenumber{1080}{af70}
\createlinenumber{1081}{aba4}
\createlinenumber{1082}{95cf}
\createlinenumber{1083}{109c}
\createlinenumber{1084}{95cf}
\createlinenumber{1085}{95cf}
\createlinenumber{1086}{95cf}
\createlinenumber{1087}{329b}
\createlinenumber{1088}{427e}
\begin{lstlisting}[language=C++]
#include <vector>
#include <utility>
#include <algorithm>
using namespace std;

struct SuffixArray {
  vector<int> sa, rank, height;

  template<typename T>
  void init(int n, const T a[]) {
    sa.resize(n);
    rank.resize(n);

    vector<pair<T, int> > assoc(n);
    for (int i = 0; i < n; ++i) {
      assoc[i] = make_pair(a[i], i);
    }
    sort(assoc.begin(), assoc.end());
    for (int i = 0; i < n; ++i) {
      sa[i] = assoc[i].second;
      if (i == 0 || assoc[i].first != assoc[i - 1].first) {
        rank[sa[i]] = i;
      } else {
        rank[sa[i]] = rank[sa[i - 1]];
      }
    }

    vector<int> tmp(n), cnt(n);
    vector<pair<int, int> > suffix(n);
    for (int m = 1; m < n; m <<= 1) {
      // snd
      for (int i = 0; i < m; ++i) {
        tmp[i] = n - m + i;
      }
      for (int i = 0, j = m; i < n; ++i) {
        if (sa[i] >= m) {
          tmp[j++] = sa[i] - m;
        }
      }
      // fst
      fill(cnt.begin(), cnt.end(), 0);
      for (int i = 0; i < n; ++i) {
        ++cnt[rank[i]];
      }
      partial_sum(cnt.begin(), cnt.end(), cnt.begin());
      for (int i = n - 1; i >= 0; --i) {
        sa[--cnt[rank[tmp[i]]]] = tmp[i];
      }
      //
      for (int i = 0; i < n; ++i) {
        suffix[i] = make_pair(rank[i],
          i + m < n ? rank[i + m] : numeric_limits<int>::min());
      }
      for (int i = 0; i < n; ++i) {
        if (i == 0 || suffix[sa[i]] != suffix[sa[i - 1]]) {
          rank[sa[i]] = i;
        } else {
          rank[sa[i]] = rank[sa[i - 1]];
        }
      }
    }

    height.resize(n);
    for (int i = 0, z = 0; i < n; ++i) {
      if (rank[i] == 0) {
        height[0] = z = 0;
      } else {
        int x = i, y = sa[rank[i] - 1];
        z = max(0, z - 1);
        while (x + z < n && y + z < n && a[x + z] == a[y + z]) {
          ++z;
        }
        height[rank[i]] = z;
      }
    }
  }
};

\end{lstlisting}
\subsection{LU分解}
\createlinenumber{1089}{6e2d}
\createlinenumber{1090}{c726}
\createlinenumber{1091}{427e}
\createlinenumber{1092}{1806}
\createlinenumber{1093}{6c2f}
\createlinenumber{1094}{178d}
\createlinenumber{1095}{95cf}
\createlinenumber{1096}{cde9}
\createlinenumber{1097}{56c5}
\createlinenumber{1098}{1bf2}
\createlinenumber{1099}{4636}
\createlinenumber{1100}{b564}
\createlinenumber{1101}{0fc0}
\createlinenumber{1102}{f3bf}
\createlinenumber{1103}{95cf}
\createlinenumber{1104}{95cf}
\createlinenumber{1105}{95cf}
\createlinenumber{1106}{9478}
\createlinenumber{1107}{e7f0}
\createlinenumber{1108}{6c2f}
\createlinenumber{1109}{6fea}
\createlinenumber{1110}{95cf}
\createlinenumber{1111}{44fd}
\createlinenumber{1112}{50eb}
\createlinenumber{1113}{95cf}
\createlinenumber{1114}{b775}
\createlinenumber{1115}{b333}
\createlinenumber{1116}{95cf}
\createlinenumber{1117}{c190}
\createlinenumber{1118}{a9f2}
\createlinenumber{1119}{3bee}
\createlinenumber{1120}{56e2}
\createlinenumber{1121}{95cf}
\createlinenumber{1122}{95cf}
\createlinenumber{1123}{95cf}
\createlinenumber{1124}{95cf}
\createlinenumber{1125}{427e}
\createlinenumber{1126}{c4d2}
\createlinenumber{1127}{5bc8}
\createlinenumber{1128}{6c2f}
\createlinenumber{1129}{2c68}
\createlinenumber{1130}{95cf}
\createlinenumber{1131}{6c2f}
\createlinenumber{1132}{1226}
\createlinenumber{1133}{99d9}
\createlinenumber{1134}{95cf}
\createlinenumber{1135}{95cf}
\createlinenumber{1136}{1894}
\createlinenumber{1137}{34a7}
\createlinenumber{1138}{99d9}
\createlinenumber{1139}{95cf}
\createlinenumber{1140}{4459}
\createlinenumber{1141}{7360}
\createlinenumber{1142}{95cf}
\createlinenumber{1143}{95cf}
\createlinenumber{1144}{6c2f}
\createlinenumber{1145}{2a61}
\createlinenumber{1146}{95cf}
\createlinenumber{1147}{95cf}
\createlinenumber{1148}{427e}
\createlinenumber{1149}{427e}
\createlinenumber{1150}{427e}
\createlinenumber{1151}{427e}
\begin{lstlisting}[language=C++]
const int MAXN = 128;
const double EPS = 1e-10;

void LU(int n, double a[MAXN][MAXN], int r[MAXN], int c[MAXN]) {
  for (int i = 0; i < n; ++i) {
    r[i] = c[i] = i;
  }
  for (int k = 0; k < n; ++k) {
    int ii = k, jj = k;
    for (int i = k; i < n; ++i) {
      for (int j = k; j < n; ++j) {
        if (fabs(a[i][j]) > fabs(a[ii][jj])) {
          ii = i;
          jj = j;
        }
      }
    }
    swap(r[k], r[ii]);
    swap(c[k], c[jj]);
    for (int i = 0; i < n; ++i) {
      swap(a[i][k], a[i][jj]);
    }
    for (int j = 0; j < n; ++j) {
      swap(a[k][j], a[ii][j]);
    }
    if (fabs(a[k][k]) < EPS) {
      continue;
    }
    for (int i = k + 1; i < n; ++i) {
      a[i][k] = a[i][k] / a[k][k];
      for (int j = k + 1; j < n; ++j) {
        a[i][j] -= a[i][k] * a[k][j];
      }
    }
  }
}

void solve(int n, double a[MAXN][MAXN], int r[MAXN], int c[MAXN], double b[MAXN]) {
  static double x[MAXN];
  for (int i = 0; i < n; ++i) {
    x[i] = b[r[i]];
  }
  for (int i = 0; i < n; ++i) {
    for (int j = 0; j < i; ++j) {
      x[i] -= a[i][j] * x[j];
    }
  }
  for (int i = n - 1; i >= 0; --i) {
    for (int j = n - 1; j > i; --j) {
      x[i] -= a[i][j] * x[j];
    }
    if (fabs(a[i][i]) >= EPS) {
      x[i] /= a[i][i];
    } // else assert(fabs(x[i]) < EPS);
  }
  for (int i = 0; i < n; ++i) {
    b[c[i]] = x[i];
  }
}

// LU(n - 1, a, r, c);
// solve(n - 1, a, r, c, b);

\end{lstlisting}
